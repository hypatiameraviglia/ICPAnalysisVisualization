%background.tex
\section{Background and Data}
The taxonomy of supernovae is, like that of planets, a system which has developed over many years as we creep intellectually through the cosmos. It is not based directly on progenitors, which we still can only feel towards in some cases, nor on explosion mechanism, the specifics of which continue to elude us. Instead, supernovae are categorized by what is simple to observe from our rock: their spectra.

First supernovae are split into Type I and Type II supernovae. Type I supernovae do not contain hydrogen -- that is, we do not observe H's spectral lines in their photometry. Type II supernovae do exhibit H spectral lines.

Type I is further divided into Type Ia, Ib, and Ic. Type Ia supernovae have silicon spectral lines. Type Ib has no silicon or hydrogen, but helium. Type Ic has no silicon, no hydrogen, and no helium at all. Type II is divided into Type IIn, IIP, IIL, and IIb. Type IIn exhibits hydrogen, as all Type II's do, but their lines are narrowed, indicating that the hydrogen was moving slower than normal. Astronomers argue this is caused by the ejection of the hydrogen before explosion, which has been substantiated by some observation. Type IIP supernovae don't differ in ingredients, but they remain at their peak (plateau) before a gradual $^56$Ni-controlled descent longer than other supernovae. Type IIL have the normal ingredients, but drop rapidly and linearly after peak. Type IIb spectra change fundamentally over time: early in the explosion hydrogen can be observed, but later these spectral lines shrink and vanish from the spectra.

Spectra are our first and dearest friends in astronomical observation. To deduce how supernovae explode and what they explode from, lightcurves provide essential information to constrain computer models of the explosion. With this interest. we compare the pre-plateau lightcurves of two supernovae: 2014J, a Type Ia, and 1987A, a Type IIP. We plot magnitude vs. time for each band individually, all bands together, and compare linear fits for the rise from cooling trough to brightness plateau.
