%TypeIIb.tex
\section{TypeII-b Example: 1987A}

Type II-P supernovae starkly contrast with the quick rise-and-drop in magnitude of other supernovae before $^{56}$Ni beta decay takes hold of the magnitude. Type II-L supernovae exhibit lightcurves very similar to those of Type Ias, but II-Ps show unusual slow rises and relatively flat plateaus that can stretch to 100 days or more. Here we plot magnitude versus time for five bands of 1987A, the quintessential Type II-P supernova, with special attention to its pre-plateau shape.

\begin{figure}[h]
	\includegraphics[width=1.0\textwidth]{1987A_U_magvstime.png}
	\caption{Change in brightness of SN1987A across time in the U band.}
\end{figure}
\begin{figure}[h]
	\includegraphics[width=1.0\textwidth]{1987A_B_magvstime.png}
	\caption{Change in brightness of SN1987A across time in the B band.}
\end{figure}
\begin{figure}[h]
	\includegraphics[width=1.0\textwidth]{1987A_V_magvstime.png}
	\caption{Change in brightness of SN1987A across time in the V band.}
\end{figure}
\begin{figure}[h]
	\includegraphics[width=1.0\textwidth]{1987A_R_magvstime.png}
	\caption{Change in brightness of SN1987A across time in the R band.}
\end{figure}
\begin{figure}[h]
	\includegraphics[width=1.0\textwidth]{1987A_I_magvstime.png}
	\caption{Change in brightness of SN1987A aross time in the I band.}
\end{figure}
\begin{figure}[h]
	\includegraphics[width=1.0\textwidth]{1987A_all_magvstime.png}
	\caption{All bands of SN1987A together. Note the similarity in magnitude vs. time curve of each of the five bands, as opposed to the distinct differences between bands observed in 2014J.}
\end{figure}
