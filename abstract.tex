%intro.tex
\section{Abstract}
When a supernova explodes, the matter of the star is ripped apart with such force that even neutrons are ripped from protons. As the expanding cloud cools, the matter forms into $^{56}$Ni, a relatively stable atom. $^{56}$Ni decays over time into $^{56}$Co, then $^{56}$Fe. Each of step of this decay process releases gamma rays, which deposit energy into the surrounging gas, "propping up" the supernova's luminosity. 

Preceding the peak of $^{56}$Ni and thus luminosity in the supernova and after the drop in luminosity due to cooling, Type Ia supernovae like SN2014J and Type II-P supernovae like 1987A differ significantly. Here we compare example lightcurves from each category and discuss implications for how the supernovae explode and how we decide taxonomy.


\end{document}
